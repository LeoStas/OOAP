% После процента идут комментарии

% Класс документа
\documentclass[11pt,a4paper,twoside,listtotoc,bibtotoc]{report}
% Размеры полей
\usepackage[top=1in, bottom=1.3in, left=1in, right=1in]{geometry}

% Кодировка файла
\usepackage[utf8]{inputenc}
% Переносы для русского языка
\usepackage[russian]{babel}


\usepackage{graphicx}
\usepackage{tikz}
\usepackage{epstopdf}
\usepackage[unicode]{hyperref}
\usepackage{float}

\usepackage{longtable}


\hypersetup{
    colorlinks,
    breaklinks,
    linkcolor=[rgb]{0.07, 0.03, 0.56},
    citecolor=[rgb]{0.1, 0.45, 0.27},
    %colorlinks=true,% hyperlinks will be coloured
    urlcolor=[rgb]{0.07, 0.03, 0.56},
    %linkbordercolor=[rgb]{0.07, 0.03, 0.56}
}


\usepackage{fancyhdr}
\fancyhead{}
\fancyhead[LE]{\thepage}
\fancyhead[CE]{{\normalfont \leftmark}}
\fancyhead[CO]{{\normalfont \rightmark}}
\fancyhead[RO]{\thepage}
\fancyfoot{}

\makeatletter
\let\ps@plain\ps@empty
\makeatother

% Отступ абзаца после заголовка
%\usepackage{indentfirst}

%\usepackage{titlesec}
\usepackage[toctitles]{titlesec}
\renewcommand{\thesection}{\arabic{section}}
% Добавить символ параграфа в нумерации section
\let\oldthesection\thesection \renewcommand\thesection{\S \oldthesection}

\usepackage{amsmath}
% Нумерация в пределах параграфа
\numberwithin{equation}{section}
% Изменить формат нумерации формул с (1.2) на (2)
\renewcommand{\theequation}{\arabic{equation}}

\usepackage{cmap}
% Разные математические шрифты
% Например, для обозначения множества вещественных чисел
\usepackage{amsfonts}
% Математические символы
\usepackage{amssymb}
% Для изменения символов нумерации
\usepackage{enumerate}
% Для нумерованных списков
\usepackage{enumitem}
% Декартов крест
\DeclareFontFamily{U}{mathx}{\hyphenchar\font45}
\DeclareFontShape{U}{mathx}{m}{n}{
      <5> <6> <7> <8> <9> <10>
      <10.95> <12> <14.4> <17.28> <20.74> <24.88>
      mathx10
      }{}
\DeclareSymbolFont{mathx}{U}{mathx}{m}{n}
\DeclareMathSymbol{\bigtimes}{1}{mathx}{"91}

\usepackage{array}

% Теоремы
\usepackage{amsthm}
    \newtheorem*{statement}{Утверждение}
    \newtheorem*{problem}{Задача}
    \newtheorem*{definition}{Определение}

    \theoremstyle{definition}
    \newtheorem{example}{Пример}[section]
    \renewcommand{\theexample}{\arabic{example}}
    \newtheorem*{example*}{Пример}

    \theoremstyle{plain}
    \newtheorem{theorem}{Теорема}[section]
    \renewcommand{\thetheorem}{\arabic{theorem}}
    \newtheorem*{theorem*}{Теорема}

    \newtheorem{new_note}{Замечание}[section]

    \newtheorem{note}{Замечание}[section]
    \renewcommand{\thenote}{\arabic{note}}



    \newtheorem{lemma}{Лемма}[section]
    \renewcommand{\thelemma}{\arabic{lemma}}
    \newtheorem*{lemma*}{Лемма}

    \newtheorem*{note*}{Замечание}

    \newtheorem{formulation}{Постановка задачи}

    \newtheorem*{formulation*}{Постановка задачи}

    \newtheorem{cons}{Следствие}[section]
    \renewcommand{\thecons}{\arabic{cons}}
    \newtheorem*{cons*}{Следствие}

    \makeatletter \renewenvironment{proof}[1][\proofname] {\par\pushQED{\qed}\normalfont\topsep6\p@\@plus6\p@\relax\trivlist\item
    [\hskip\labelsep\bfseries#1\@addpunct{.}]\ignorespaces}
    {\popQED\endtrivlist\@endpefalse} \makeatother

    \makeatletter \newenvironment{solution}[1][Решение] {\par\pushQED{\qed}\normalfont\topsep6\p@\@plus6\p@\relax\trivlist\item
    [\hskip\labelsep\bfseries#1\@addpunct{.}]\ignorespaces}
    {\popQED\endtrivlist\@endpefalse} \makeatother

% Новая строка после названия пункта
\makeatletter
\renewcommand\paragraph{%
   \@startsection{paragraph}{4}{0mm}%
      {-\baselineskip}%
      {.5\baselineskip}%
      {\normalfont\large\bfseries}}
\makeatother

\makeatletter
\DeclareRobustCommand*{\bfseries}{%
  \not@math@alphabet\bfseries\mathbf
  \fontseries\bfdefault\selectfont
  \boldmath
}
\makeatother

% Для систем уравнений
\usepackage{cases}

\makeatletter
\newenvironment{sqcases}{%
  \matrix@check\sqcases\env@sqcases
}{%
  \endarray\right.%
}
\def\env@sqcases{%
  \let\@ifnextchar\new@ifnextchar
  \left\lbrack
  \def\arraystretch{1.2}%
  \array{@{}l@{\quad}l@{}}%
}
\makeatother

\usepackage{mathtools}
% Норма вектора
\DeclarePairedDelimiter\norm{\lVert}{\rVert}
% Целая часть сверху
\DeclarePairedDelimiter{\ceil}{\lceil}{\rceil}
% Оператор sgn
\DeclareMathOperator{\sgn}{sgn}
% Оператор diag
\DeclareMathOperator{\diag}{diag}
%\renewcommand\labelitemi{---}
%\fancyhead{}
%\fancyhead[LE,RO]{\thepage}
%\fancyhead[RE]{\leftmark}
%\fancyhead[LO]{\rightmark}
%
%
%\setcounter{secnumdepth}{1}
%\setcounter{tocdepth}{1}
%\newcommand{\bigO}[1]{\ensuremath{\operatorname{O}\bigl(#1\bigr)}}
%

\makeatletter
\DeclareRobustCommand*{\bfseries}{%
  \not@math@alphabet\bfseries\mathbf
  \fontseries\bfdefault\selectfont
  \boldmath
}
\makeatother

\begin{document}
\pagestyle{empty}
%
\begin{center}
    \vspace*{-0.7in}
    \includegraphics[width=0.5\textwidth]{msu.eps}\\
    {\scshape Московский государственный университет имени М.\,В. Ломоносова}\\
    Факультет вычислительной математики и кибернетики\\
    Кафедра АСВК\\
    \vfill
    %{\LARGE }\\
    %\vspace{0.55cm}
    {\Huge\bfseries Отчет по второму заданию курса ООАП.}\\
    \vspace{0.8cm}
    {\LARGE Вариант №$4$}\\
    {\LARGE <<Онлайновая билетная касса>>}

\end{center}
%
\vspace{6cm}
\begin{flushright}
    \large
    \textit{Выполнил:}\\
    Селецкий С.\,В., $421$ группа\\
\end{flushright}
%
\vfill
%
\begin{center}
    Москва, 2013
\end{center}
%
%
\newpage
%
\addtocontents{toc}{\protect\thispagestyle{empty}}
\tableofcontents
\newpage
\phantomsection
\pagestyle{plain}
%
\section{Постановка задачи}
%
Онлайновая касса <<чух-чух.рф>> представляет собой web-сайт службы
бронирования и доставки билетов на поезда дальнего следования.
Перед тем, как впервые воспользоваться услугами кассы, клиент должен
зарегистрироваться. В ходе регистрации он указывает данные о себе
(фио, телефон, адрес электронной почты, паспортные данные) и получает
логин и пароль (логины и пароли разных клиентов не должны совпадать).

Войдя в систему, клиент может ознакомиться с расписанием движения
поездов между нужными ему населёнными пунктами, выбрать станцию и
дату отправления, а также номер поезда. Получив от системы сведения
о билетах, имеющихся в наличии, пользователь может забронировать
нужное ему количество билетов. Билеты бывают разных типов: плацкартные,
купейные ($1$, $2$ и $3$ класса), сидячие, в мягкий вагон, в вагон люкс
и т. п. Места могут быть верхними, нижними, боковыми, в зависимости от
расположения в вагоне. Цена билета зависит от маршрута следования,
типа поезда (фирменный/скорый/пассажирский), типа вагона, в котором расположено
пассажирское место, и расположения места в вагоне. Билеты могут быть
выкуплены в течение трех суток с момента бронирования, но не позднее
двух суток до отправления поезда. Клиент может самостоятельно выкупить
забронированные билеты, приехав в офис, или заказать доставку билетов курьером,
сделав пометку в заявке и указав адрес доставки. Стоимость доставки зависит от
дальности: центр/спальный район/дальний пригород. Клиент может получить
информацию обо всех своих заявках с web-страницы онлайновой кассы.

Заявки клиентов хранятся в системе. В каждой указаны: сведения о клиенте,
сведения об одном или более билетах (к одной заявке может относиться
несколько билетов на одно и то же имя), общая стоимость билетов в
заявке, время создания заявки, время оплаты, вид доставки
(самовывоз/курьер), адрес доставки, стоимость доставки, статус заявки
(новая/рабочая/оплаченная/аннулированная). По истечении $12$ месяцев с
момента создания заявки данные автоматически удаляются из системы.

В обязанности работников онлайновой кассы входит внесение в систему сведений
о расписании поездов, стоимости проезда и об имеющихся в наличии местах, на
которые могут быть куплены билеты. Некоторые поезда ходят ежедневно. Другие
назначаются только в определённый день (определённые дни) недели. Также есть
отдельные поезда, назначаемые на конкретные дни года. Данные о билете содержат
уникальный номер билета, даты и станции отправления и прибытия, тип билета,
номер места, цену билета, статус билета
(место свободно/забронировано/продано/передано для реализации в кассы на вокзалах).
По истечении 12 месяцев
с даты, указанной в билете, данные автоматически удаляются из системы.

Работник кассы, получив новую заявку клиента, связывается с ним для подтверждения
и уточнения мест. Согласовав с клиентом места, работник делает пометку о бронировании
билетов в системе (тем самым уменьшается количество билетов, имеющихся в наличии)
и меняет статус заявки на <<рабочая>>. После оплаты и/или доставки <<рабочей>> заявки
билеты из заявки помечаются как проданные, а заявка --- как оплаченная. За $2$ суток
до отправления поезда все непроданные билеты передаются для реализации в обычные
кассы, в системе они автоматически помечаются как <<передан для реализации>>,
заявки на них аннулируются, клиенты, не успевшие оплатить заказанные билеты,
информируются о снятии брони. Через $4$ суток после создания все неоплаченные <<рабочие>>
заявки автоматически аннулируются, бронирование с билетов снимается, клиентам
посылается соответствующее сообщение. Также должна быть возможность аннулирования
заявок вручную работниками онлайновой кассы. При аннулировании заявки вручную работник
должен уведомить клиента, изменить статус заявки, снять бронирование билетов
(количество билетов в наличии возрастает).

%
\newpage
%
%
\section{Глоссарий}
%
%
\begin{center}
%
\begin{longtable}[h!]{ | p{0.3\linewidth} || p{0.7\linewidth} |}
%
    \hline
    
    {\bfseries Система} $(System)$ & Система службы бронирования и доставки билетов, предоставляющая услуги клиентам. \\
    \hline
    
    {\bfseries Клиент} $(Client)$ & Лицо, пользующееся услугами системы. \\
    \hline
    
    {\bfseries База данных клиентов} $(ClientsDB)$
        &
        База данных, содержащая информацию о клиентах, вносимую ими при регистрации (ФИО, 
        номер телефона, адрес электронной почты, паспортные данные). А также автоматически 
        генерируемую информацию (уникальный логин, пароль).\\
    \hline
    
    {\bfseries Расписание} $(Schedule)$ & Содержит всю информацию о движении поездов между станциями
        (номера поездов, которые останавливаются на этих станциях; стоимость проезда на каждом поезде;
        дата и время прибытия каждого поезда на станции и отправки с них).
    \\
    \hline

    {\bfseries Место} $(Seat)$ & Место в поезде (тип места (верхнее, нижнее или боковое); номер места в вагоне;
        номер вагона в поезде).
     \\
    \hline

    {\bfseries Поезд} $(Train)$ & Хранит информацию о поезде (номер поезда; тип поезда (фирменный~/ скорый~/ пассажирский);
            для каждой станции хранится список билетов, действующих на этой станции; список мест в поезде;
            список посещаемых станций; времена и даты прибытия и отправки от каждой станции).
    \\
    \hline

    {\bfseries Список поездов} $(TrainsList)$ & Список всех экземпляров типа <<Поезд>>. \\
    \hline

    {\bfseries Билет} $(Ticket)$ & Содержит информацию о билете (уникальный номер билета; дата и время отправления и прибытия;
            станции отправления и прибытия; уникальный номер места (т.е. тройка: номер поезда, номер вагона, номер места в вагоне);
            тип билета (плацкартный, купейный ($1$, $2$ и $3$ класса), сидячий, в мягкий вагон, в вагон люкс и т. п.); 
            цена билета (зависит от стоимости маршрута следования, типа поезда, типа билета и типа места); 
            статус билета (место свободно~/ забронировано~/ продано~/ передано для реализации в кассы на вокзалах);
            данные клиента, купившего или забронировавшего билет (если место продано или забронировано).
    \\
    \hline
    
    {\bfseries База данных всех билетов} $(TicketsDB)$ & Содержит информацию о всех билетах в системе. \\
    \hline

    {\bfseries Имеющиеся в наличии билеты} $(AvailableTickets)$ & Хранит информацию о доступных для продажи в системе билетах. \\
    \hline

    {\bfseries База данных заявок} $(RequestsDB)$ & Все заявки, хранимые в системе. \\
    \hline

    {\bfseries Заявка} $(Request)$ & Хранит информацию о заявке клиента (сведения о клиенте; сведения об одном и более билетах 
        (к одной заявке может относиться несколько билетов на одно и то же имя); общая стоимость билетов в заявке; 
        время создания заявки; время оплаты; вид доставки (самовывоз/курьер); адрес доставки; стоимость доставки; статус заявки 
        (новая~/ рабочая~/ оплаченная~/ аннулированная).
    \\
    \hline

    {\bfseries Работники онлайновой кассы} $(Staff)$ & В обязанности работников онлайновой кассы входит внесение в систему сведений
        о расписании поездов, стоимости проезда и об имеющихся в наличии местах, на которые могут быть куплены билеты. \\
    \hline

    {\bfseries Работники службы доставки} $(DeliveryWorkers)$ & В их обязанности входит доставка билетов до клиентов. \\
    \hline

\end{longtable}
%
\end{center}
%
%
\newpage
%
%
\section{Варианты использования}
%
%
\newpage
%


\end{document} 
